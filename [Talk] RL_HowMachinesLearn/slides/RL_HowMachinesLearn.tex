%-------------------------------------------------------
%-- PREAMBLE
%-------------------------------------------------------
\documentclass[8pt]{beamer}
\usetheme[]{Feather}
  
\setbeamersize{text margin left=0.75cm,text margin right=0.75cm}

% INCLUDE PACKAGES
%-------------------------------------------------------

\usepackage[utf8]{inputenc}
\usepackage[english]{babel}
\usepackage[T1]{fontenc}
\usepackage{helvet}

\usepackage{graphicx}
\graphicspath{ {imgs/} }

\usepackage[utf8]{inputenc}
\usepackage[english]{babel}
\usepackage[protrusion=true,expansion=true]{microtype} 
\usepackage{amsmath,amsfonts,amsthm,amssymb,bm,mathdots,mathtools,bigints}
\usepackage{color, xcolor}
\usepackage{listings}
\usepackage[document]{ragged2e}
\usepackage{wrapfig}
\usepackage[printwatermark]{xwatermark}
\usepackage{subcaption}
\usepackage{mdframed}
\usepackage{multicol}
\usepackage{environ}
\usepackage{tikz, pgfplots}
\usepackage{framed}


\lstset{
    backgroundcolor=\color[rgb]{0.86,0.88,0.93},
    language=matlab, keywordstyle=\color[rgb]{0,0,1},
    basicstyle=\footnotesize \ttfamily,breaklines=true,
    escapeinside={\%*}{*)}
}

% DEFFINING COLORS
%-------------------------------------------------------
\definecolor{glgRed}{RGB}{214,45,32}
\definecolor{glgGreen}{RGB}{0,135,68}
\definecolor{glgBlue}{RGB}{0,87,231}
\definecolor{glgOrange}{RGB}{255,167,0}

\definecolor{retroBrown}{RGB}{102,101,71}
\definecolor{retroRed}{RGB}{251,46,1}
\definecolor{retroGreen}{RGB}{111,203,159}
\definecolor{retroYellow}{RGB}{255,226,138}

\definecolor{pastelBlue}{RGB}{27,133,184}
\definecolor{pastelRed}{RGB}{174,90,65}
\definecolor{pastelGreen}{RGB}{85,158,131}
\definecolor{pastelBlack}{RGB}{90,82,85}

\definecolor{niceBlack}{RGB}{14,17,17}
\definecolor{niceBlue}{RGB}{14,104,206}

\definecolor{monokaiBG}{RGB}{39,40,34}
\definecolor{monokaiOrange}{RGB}{253,151,31}

\colorlet{signal1}{niceBlack!90}
\colorlet{signal2}{niceBlack!90}
\colorlet{signal3}{niceBlack!90}
\colorlet{signal4}{niceBlack!90}

% Beamer Color Scheme
\setbeamercolor{Feather}{fg=niceBlack!50, bg=monokaiBG}
\setbeamercolor{Feathertext}{fg=monokaiOrange}
\setbeamercolor{structure}{fg=monokaiOrange!95!niceBlack}
\setbeamercolor{frametitle}{bg=monokaiBG!80}
\setbeamercolor{normal text}{fg=black!85}


% Tikz Macros
% --------------------------------------------------------------------
\usetikzlibrary{shapes,arrows, backgrounds, positioning, fit, decorations.pathmorphing}

\tikzstyle{sectionBlock} = [draw, fill=blue!20, rectangle, minimum height=3em, minimum width=3em]
\tikzstyle{block} = [draw, fill=white, rectangle, minimum height=3em, minimum width=3em]
\tikzstyle{sum} = [draw, fill=white, circle, node distance=1cm]
\tikzstyle{input} = [coordinate]
\tikzstyle{output} = [coordinate]
\tikzstyle{pinstyle} = [pin edge={to-,thin,black}]
\tikzstyle{mcCircle} = [draw, circle, line width=0.5mm, minimum width=3em, minimum height=3em, fill=white]
\tikzstyle{mcInput} = [draw, rectangle, line width=0.5mm, minimum width=3em, minimum height=3em, fill=black!10]
\tikzstyle{mcCircleX} = [draw=glgBlue!80, circle, line width=0.75mm, minimum width=3em, minimum height=3em, fill=white]
\tikzstyle{mcCircleY} = [draw=glgBlue!80, circle, line width=0.75mm, minimum width=3em, minimum height=3em, fill=white]


% DEFFINING AND REDEFINING COMMANDS
%-------------------------------------------------------

% colored hyperlinks
\newcommand{\chref}[2]{
  \href{#1}{{\usebeamercolor[bg]{Feather}#2}}
}

\makeatletter
\let\beamer@writeslidentry@miniframeson=\beamer@writeslidentry
\def\beamer@writeslidentry@miniframesoff{%
  \expandafter\beamer@ifempty\expandafter{\beamer@framestartpage}{}% does not happen normally
  {%else
    % removed \addtocontents commands
    \clearpage\beamer@notesactions%
  }
}
\newcommand*{\miniframeson}{\let\beamer@writeslidentry=\beamer@writeslidentry@miniframeson}
\newcommand*{\miniframesoff}{\let\beamer@writeslidentry=\beamer@writeslidentry@miniframesoff}
\makeatother

\makeatletter
\patchcmd{\beamer@sectionintoc}
  {\vfill}
  {\vskip\itemsep}
  {}
  {}
\makeatother  

% tensor 2:
\newcommand{\tend}[1]{\hbox{\oalign{$\bm{#1}$\crcr\hidewidth$\scriptscriptstyle\bm{\sim}$\hidewidth}}}
\newcommand{\tenq}[1]{\hbox{\oalign{$\bm{#1}$\crcr\hidewidth$\scriptscriptstyle\bm{\sim}$\hidewidth}}}


% ENVIROMENTS
%-------------------------------------------------------

\AtBeginSection[]{
  \begin{frame}
  \vfill
    \centering
    \bfseries \Huge \color{monokaiOrange!85} \insertsectionhead\par%
  \vfill
  \end{frame}
}

%\AtBeginSubsection[]{
%  \begin{frame}
%  \vfill
%    \centering
%    \Huge \color{glgBlue!60} \insertsubsectionhead\par%
%    \huge \color{niceBlack!85} \insertsectionhead\par%
%  \vfill
%  \end{frame}
%}

\newenvironment<>{varblock}[2][.9\textwidth]{%
  \setlength{\textwidth}{#1}
  \begin{actionenv}#3%
    \def\insertblocktitle{#2}%
    \par%
    \usebeamertemplate{block begin}}
  {\par%
    \usebeamertemplate{block end}%
  \end{actionenv}}

\NewEnviron{myequation*}{%
    \begin{equation*}
    \scalebox{1.1}{$\BODY$}
    \end{equation*}
    }
    
% INFORMATION IN THE TITLE PAGE
%-------------------------------------------------------

\title[Reinforcement Learning: Como as máquinas aprendem?]
{
      Reinforcement Learning:\\Como as máquinas aprendem?
}

\subtitle[]
{
}

\author[Minho Menezes]
{      
  Minho Menezes\\
}

\institute[]
{
  EEEP Joaquim Nogueira - Fortaleza\\
  
  %there must be an empty line above this line - otherwise some unwanted space is added between the university and the country (I do not know why ;( )
}

\date{\today}

%-------------------------------------------------------
%-- THE BODY OF THE PRESENTATION
%-------------------------------------------------------

\begin{document}

%-------------------------------------------------------
% THE TITLEPAGE
%-------------------------------------------------------

{ \1
\begin{frame}[plain,noframenumbering] % the plain option removes the header from the title page, noframenumbering removes the numbering of this frame only
  \titlepage % call the title page information from above
\end{frame}}

%-------------------------------------------------------
% SUMMARY
%-------------------------------------------------------

\begin{frame}
  \setcounter{secnumdepth}{1}
  \setcounter{tocdepth}{1}
  \begin{center} \Large \bfseries
    Summary
  \end{center}
  \vfill
    \begin{columns}[t]
    	\hfill
        \begin{column}{0.4\textwidth}
        		\bfseries \large
            \tableofcontents[sections={1-3}]
        \end{column}
        \begin{column}{0.4\textwidth}
            \tableofcontents[sections={4-6}]
        \end{column}
        \hfill
    \end{columns}
\end{frame}


%-------------------------------------------------------
% SECTION - Introduction
%-------------------------------------------------------
\section{Introduction}
%-------------------------------------------------------
\subsection{Motivation}
%-------------------------------------------------------
\begin{frame}[c]{Introduction}{Contextualization}

\vskip0.25cm

\begin{itemize} \justifying
  \item<1-> We are discussing the \textbf{optimal control} of \textbf{dynamical systems}. \vskip0.2cm
  
  \item<2-> We are discussing \textbf{chemical reaction network systems}, for processes described as:\vskip0.2cm
  
\end{itemize}

\end{frame}

%-------------------------------------------------------
\subsection{Problem Definition}
%-------------------------------------------------------
\begin{frame}[t]{Introduction}{Problem Definition}

\begin{center}
\begin{varblock}[1\textwidth]{} \centering

\end{varblock}
\end{center} 

\end{frame}

%-------------------------------------------------------
\begin{frame}[t]{Introduction}{Problem Definition}
\addtocounter{footnote}{-1}

\vskip0.25cm

\begin{center}
\begin{varblock}[1\textwidth]{Text} \centering
  \begin{minipage}{.5\textwidth}
    {\justifying \bfseries This system...} \vskip0.2cm
    
    \begin{itemize} \justifying
      \item class of system that represents a wide range of industrial applications.\vskip0.2cm
      
      \item classical benchmark for multiple-input multiple-output (MIMO) control systems.\vskip0.2cm
      
      \item nonlinear behavior, models with non-minimum phase behavior and unmeasurable states.\vskip0.2cm
    \end{itemize}
  \end{minipage}
\end{varblock}
\end{center}

\end{frame}


%-------------------------------------------------------
%-- End Page
{\1
\begin{frame}[plain,noframenumbering]
  \center \Huge \textbf{\color{monokaiOrange} Thank you!}       
  
  \includegraphics[scale=0.4]{corgi.jpg}
  
  {\color{monokaiOrange} Questions?}       
\end{frame}

%-------------------------------------------------------
\end{document}


%%---------------------Rascunho--------------------------
%\section{Titulo}
%\subsection{Subtitulo}
%%---------------------Rascunho--------------------------
%\begin{frame}{Titulo}{Subtitulo}
%
%\begin{block}{title}
% Say somethings \alert{new} 
%\end{block}
%
%\begin{itemize}
% \item TikZ\footnote{TikZ is a package for creating beautiful graphics. Have a look at these \chref{http://www.texample.net/tikz/examples/}{online examples} or the \chref{http://tug.ctan.org/tex-archive/graphics/pgf/base/doc/generic/pgf/pgfmanual.pdf}{pgf user manual}.}
%\end{itemize}
%
%\end{frame}
%%---------------------Rascunho--------------------------

%%%%%%%%%%%%%%%%%%%%%%%%%%%%%%%%%%%
%% Drafts:
%%%%%%%%%%%%%%%%%%%%%%%%%%%%%%%%%%%
%%%%% Figure:
% \begin{figure}[ht]
%   \centering
%   \includegraphics[trim={0cm 0cm 0cm 0cm},clip,scale=1]{nameFigure}
%   \caption{caption of the figure.}
%   \label{fig:nameFigure}
% \end{figure} \vskip0.25cm
%
%%%%% Equation:
% \begin{equation*} \label{eq:nameequation*}
% \begin{split}
%  X = 1 + 1
% \end{split}
% \end{equation*} \vskip0.25cm
%
%%%% Table:
% \begin{table}[hp]
%   \centering
%   \begin{tabular}{l | c c }
%   Principal & Coluna1 & Coluna2 \\
%   \hline 
%   ABC & 1 & 2 \\
%   DFG & 3 & 4 \\
%   HIJ & 5 & 6 \\
%   \end{tabular} 
%   \caption{caption of the table.}
%   \label{table:nameTable} 
% \end{table} \vskip0.25cm
